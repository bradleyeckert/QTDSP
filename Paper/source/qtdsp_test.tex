\section{Testing}

The demodulation algorithm was tested by coding the algorithm in C (as a console
application) and instrumenting it to display variables,
save arrays to files, and benchmark the various stages.
KissFFT is used as the 1K complex FFT.
On a fast Core-i7 PC, the algorithm correlates a frame in about 100 usec.
Allowing 100 usec per 32 points amounts to 320K SPS of input which, when $R=-0.5$,
is 80K SPS of output.

\subsection{EEG data}

An ideal test for the algorithm is EEG data taken from public data sets.
https://www.physionet.org/pn6/chbmit/ has EEG datasets with epileptic seizures
in edf format.

The DREAMS Sleep Spindles Database at Facult\'e Polytechnique de Mons (Belgium)
TCTS Lab \cite{Devuyst} provides EEG data in text format. 
The ``excerpt2'' data set consists of 30 minutes of 200 SPS data as well as the
stage of sleep for each 5-second interval.
Sleep is useful to look at due to the presence of alpha, delta and theta waves. 
The session has all stages of sleep from awake to REM to stage 4.
If exponential chirp spectra are present and forming informational pulse trains,
that's an indication that the brain is a receiver where consciousness is
experienced. The origin of consciousness is in a timeless, eternal realm.

The demodulation algorithm looks at the signal bandwidth between about
$F_S/5$ and $F_S/2$.
Many data sets have a region of interest far below the sample rate.

Decimation is typically used to lower the sample rate to bring it to the
desired region of interest. 
Decimation by a factor of M usually involves low-pass filtering the signal
and taking every Mth data point as output.
A more sloppy method is to simply sum each set of M input points and ignore
the aliasing effects since those will be smeared across the noise floor.
The spectrogram of ``excerpt2'' data shows little information above 30 Hz.
The 200 Hz data is downsampled to 50 or 67 Hz.

The demodulation app processes up to a minute or so of data starting at a 
beginning time and saves it as a BMP (using colored amplitude) with time on the
horizontal axis and R on the vertical.
It's feasible to implement just a console application,
avoiding GUI work altogether.
The BMP is the scientific output, viewable in many media as well as on the web.
For an R range of 0.25 to 0.75, there are between 3 and 8 input points per 
output point. 

The spectrum is presented in polar format to account for the amount of data in
each row (or arc) being proportional to the R value.
Radius represents R and angle represents time.

